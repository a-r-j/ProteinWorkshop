\documentclass{article}

% if you need to pass options to natbib, use, e.g.:
%     \PassOptionsToPackage{numbers, compress}{natbib}
% before loading neurips_data_2023

% ready for submission
\usepackage{neurips_data_2023}

% to compile a preprint version, add the [preprint] option, e.g.:
%     \usepackage[preprint]{neurips_data_2023}
% This will indicate that the work is currently under review.

% to compile a camera-ready version, add the [final] option, e.g.:
%     \usepackage[final]{neurips_data_2023}

% to avoid loading the natbib package, add option nonatbib:
%    \usepackage[nonatbib]{neurips_data_2023}

% Submissions to the datasets and benchmarks are typically non anonymous,
% but anonymous submissions are allowed. If you feel that you must submit 
% anonymously, you can compile an anonymous version by adding the [anonymous] 
% option, e.g.:
%     \usepackage[anonymous]{neurips_data_2023}
% This will hide all author names.

\usepackage[utf8]{inputenc} % allow utf-8 input
\usepackage[T1]{fontenc}    % use 8-bit T1 fonts
\usepackage{hyperref}       % hyperlinks
\usepackage{url}            % simple URL typesetting
\usepackage{booktabs}       % professional-quality tables
\usepackage{amsfonts}       % blackboard math symbols
\usepackage{nicefrac}       % compact symbols for 1/2, etc.
\usepackage{microtype}      % microtypography
\usepackage{xcolor}         % colors
\usepackage{adjustbox}
\bibliographystyle{unsrtnat}
\setcitestyle{numbers,open={[},close={]},citesep={,}}

%\title{Evaluating Representation Learning on Protein Structures}
\title{Evaluating Representation Learning on the Protein Universe}

% The \author macro works with any number of authors. There are two commands
% used to separate the names and addresses of multiple authors: \And and \AND.
%
% Using \And between authors leaves it to LaTeX to determine where to break the
% lines. Using \AND forces a line break at that point. So, if LaTeX puts 3 of 4
% authors names on the first line, and the last on the second line, try using
% \AND instead of \And before the third author name.

\author{%
  David S.~Hippocampus%\thanks{Use footnote for providing further information
    %about author (webpage, alternative address)---\emph{not} for acknowledging
    %funding agencies.} \\
  Department of Computer Science\\
  Cranberry-Lemon University\\
  Pittsburgh, PA 15213 \\
  \texttt{hippo@cs.cranberry-lemon.edu} \\
  % examples of more authors
  % \And
  % Coauthor \\
  % Affiliation \\
  % Address \\
  % \texttt{email} \\
  % \AND
  % Coauthor \\
  % Affiliation \\
  % Address \\
  % \texttt{email} \\
  % \And
  % Coauthor \\
  % Affiliation \\
  % Address \\
  % \texttt{email} \\
  % \And
  % Coauthor \\
  % Affiliation \\
  % Address \\
  % \texttt{email} \\
}

\begin{document}

\maketitle

\begin{abstract}
% Protein structure representation learning is the foundation for promising applications in drug discovery, protein design, and function prediction.
% However, there remains a need for a robust, standardised benchmark to track the progress of new and established methods with greater granularity and relevance to downstream applications. 
We introduce \emph{ProteinWorkshop}, a comprehensive benchmark suite for Geometric Graph Neural Networks and protein structure representation learning.
% We provide large-scale pre-training and downstream tasks comprised of both experimental and predicted structures, offering a balanced challenge to representation learning algorithms. 
% These tasks enable the systematic evaluation of the quality of the learned embeddings, the structural and functional relationships captured, and their usefulness in downstream tasks. 
We consider large-scale pre-training and downstream tasks on both experimental and predicted structures to enable the systematic evaluation of the quality of the learned structural representation, the functional relationships captured, and their usefulness in downstream tasks. 
% We benchmark state-of-the-art protein-specific and generic geometric Graph Neural Networks and the extent to which they benefit from different types of pre-training. 
% We find that pre-training consistently improves the performance of both rotation-invariant and equivariant geometric models, and that equivariant models seem to benefit even more from pre-training compared to invariant models.
We find that: (1) large-scale pretraining on AlphaFold-predicted structures and auxiliary tasks consistently improve the performance of both rotation-invariant and equivariant Geometric GNNs, and (2) more expressive equivariant GNNs benefit from pretraining to a greater extent compared to invariant models.

We aim to establish a common ground for the machine learning and computational biology communities to rigorously compare and advance protein structure representation learning. 
% By providing a standardised and rigorous evaluation platform, we expect to accelerate the development of novel methodologies and improve our understanding of protein structures and their functions. 
Our open-source codebase reduces the barrier to entry for working with large protein structure datasets by providing: (1) storage-efficient dataloaders from large-scale predicted structures including AlphaFoldDB and ESM Atlas, as well as (2) utilities for constructing new tasks from the entire PDB.
\emph{ProteinWorkshop} is available at: \href{https://github.com/a-r-j/ProteinWorkshop}{\texttt{github.com/a-r-j/ProteinWorkshop}}.

% The codebase incorporates several engineering contributions which considerably reduces the barrier to entry for pre-training and working with large structure-based datasets. Our benchmark is available at: \url{https://anonymous.4open.science/r/ProteinWorkshop-B8F5}. 
%at \href{https://www.github.com/a-r-j/ProteinWorkshop}{\texttt{https://www.github.com/a-r-j/ProteinWorkshop}}.

\end{abstract}

\section{Introduction}
Proteins play central roles in cellular processes and understanding their function is...


Developments in protein structure prediction methods have led to an explosion in the availability of structural data. This has resulted in a significant gap between meaningful annotations due to the significant. 


Several deep learning methods have been developed to learn from protein structures. 


In this work we develop a large-scale framework for developing and evaluating protein structural encoders, providing several pre-training corpuses that span known foldspace and tasks that assess the ability of models to learn informative representations at different levels of structural granularity.

Previous works in protein structure representation learning have focussed on learning effective \emph{global} (i.e. graph-level) representations of protein structure, typically evaluating the methods on function or fold classification tasks. However, there has been comparatively little investigation into the ability of different methods to learn informative local (\emph{node-level}) representations. Good node-level representations are important for a variety of annotation tasks, such as binding or interaction site prediction, as well as providing conditioning signals in structure-conditioned molecule design methods. Crucially, understanding the structure-function relationship at this granular level can drive progress in protein design by revealing structural motifs that underlie desirable properties enabling them to be incorporated into designs.



Our contributions are as follows:

\begin{enumerate}
    \item 
\end{enumerate}
\section{Related Work}

\textbf{Protein Structure Representation Learning. } Several structure-based encoders for proteins have been designed to extract information from different levels of granularity, such as residue-level, atom-level and surfaces. Previous works have aimed to encode protein structural priors directly within architectures to model proteins hierarchically \citep{somnath2021multi, hermosilla2020intrinsic}, as computationally efficient point clouds \citep{gainza2020deciphering, sverrisson2021fast}, or as geometric graphs \citep{jing2020learning, jin2021iterative, morehead2024geometry, wang2023learning, zhang2023protein, mahmud2023accurate} for tasks such as protein function prediction \citep{Gligorijevi2021}, protein model quality assessment \citep{eismann2020protein, chen20233d, morehead2024gcpnetema}, and protein interaction region prediction \citep{dai2021protein, morehead2023dips}.

\textbf{Protein Benchmarks. } Several benchmarks have been proposed for evaluating the efficacy of learnt protein \emph{sequence} representations. However, \emph{structure-based} benchmarks are comparatively unaddressed. \citet{tape} developed TAPE (Tasks Assessing Protein Embeddings), providing a large pretraining corpus of protein sequences curated from Pfam \citep{ElGebali2018}, as well as a collection of five supervised benchmark tasks assessing the ability of protein language models to predict structural qualities (contact prediction and secondary structure prediction), and functional properties (fluorescence and stability prediction). \citet{peer} proposed PEER (Protein Sequence Understanding), focussing on multitask evaluation of protein sequence models. Therapeutic Data Commons \citep{NEURIPSDATASETSANDBENCHMARKS2021_4c56ff4c} provide several datasets relevant to therapeutic development, however the few protein structure-derived datasets it contains are cast as sequence-based tasks. \citet{NEURIPSDATASETSANDBENCHMARKS2021_2b44928a} developed FLIP, a sequence-based benchmark of protein fitness landscapes. ProteinGLUE \citep{Capel2022} is another sequenced-based benchmark focussing on per-residue tasks.

To our best knowledge, the only protein structure-benchmark to date is ATOM3D \citep{NEURIPSDATASETSANDBENCHMARKS2021_c45147de}, which proposes a collection of tasks largely assessing geometric methods at predicting graph-level properties of protein structures. TorchProtein \citep{zhu2022torchdrug} also provides a small collection of global-structural datasets. Most existing benchmarks do not exhaustively evaluate both the local and global representation learning power of proposed methods. As the field develops, we identify a need for a consistent benchmarking framework of diverse tasks to ensure improving results reported in the literature map on to progress in the downstream problems we hope to address.
Similar benchmarking efforts for general purpose GNNs have provided experimental rigour to architectural research \citep{dwivedi2020benchmarking}.

\textbf{Denoising-Based Pre-training and Regularisation. } 
Several methods have been developed for pre-training GNNs, predominantly focussing on cases where 3D coordinate information is only implicitly encoded in the graph structures. In this work, we build on work by \citet{godwin2021simple} and \citet{zaidi2023pretraining} to investigate whether denoising-based auxillary and pre-training tasks are effective methods for pre-training geometric GNNs operating on protein structures, similar to concurrent works bridging the gap between denoising objectives for geometric neural networks and diffusion generative modeling for biomolecules \citep{huang2023data, corso2023modeling}.
\section{ProteinWorkshop}
The overarching goal of \emph{ProteinWorkshop} is to effectively cover the design space of protein structure representation learning methods. To achieve this, the benchmark is highly modular by design, enabling evaluation of different combinations of structural encoders, protein featurisation schemes, and auxiliary tasks over a wide range of both supervised and unsupervised tasks.
A user manual is available in Appendix \ref{app:benchmark}, containing detailed listings and descriptions of all components.

\subsection{Featurisation Schemes}
Protein structures are typically represented as geometric graphs, with researchers opting to use a coarse-grained C$\alpha$ atoms graph as full atom representations can quickly become computationally intractable due to a large number of nodes. 
However, this is a lossy representation, with much of the structural detail, such as orientation of the backbone and sidechain structure, being only implicitly encoded.
Due to the computational burden incurred by operating on full-atom node representations, we focus primarily on C$\alpha$-based graph representations, investigating featurisation strategies to incorporate higher-level structural information. 
Note that we do provide utilities to enable users to work with backbone and full-atom graphs in the benchmark.
% We represent protein structures as geometric graphs, $\mathcal{G} = (\mathcal{V}, \mathcal{E}, \mathbf{\vec{X}}, \mathbf{S}, \mathbf{\vec{V}})$, where $\mathcal{V}$ is a set of nodes, $\mathcal{E}$ is a set of edges, $\mathbf{\vec{X}} \in \mathbb{R}^{|\mathcal{V}| \times 3}$ is a matrix of Cartesian node coordinates, $\mathbf{S} \in \mathbb{R}^{|\mathcal{V}| \times d}$ is a matrix of $d$-dimension scalar node features, and $\mathbf{\vec{V}} \in \mathbb{R}^{|\mathcal{V}| \times d \times 3}$ is a tensor of vector-valued features. 
Details about different featurisation schemes are provided in Appendix \ref{app:featurisation} and Table \ref{tab:features}.

\subsection{Pre-training Tasks}

The benchmark contains a comprehensive suite of pretraining tasks. Broadly, these can be categorised into: masked-attribute prediction, denoising-based and contrastive learning-based tasks. These can be used as both a pretraining objective or as auxiliary tasks in a downstream supervised task.

\textbf{Sequence Denoising. } The benchmark contains two variations based on two sequence corruption processes $C(\tilde{\mathcal{S}} | \mathcal{S}, \nu)$ that receive an amino acid sequence $\mathcal{S} \in [0, 1]^{|\mathcal{V}| \times 23 }$ and return a sequence $\mathcal{S} \in [0, 1]^{|\mathcal{V}| \times 23 }$ with fraction $\nu$ of its positions corrupted. The first scheme is based on mutating a fraction of the residues to a random amino acid and tasking the model with recovering the uncorrupted sequence. The second is a masked residue prediction task, where a fraction of the residues are altered to a mask value and the model is tasked to recover the uncorrupted sequence.

\textbf{Structure Denoising. } We provide two structure-based denoising tasks: coordinate denoising and torsional denoising. In the coordinate denoising task, noise is sampled from a normal or uniform distribution and scaled by noise factor, $\nu \in \mathbb{R}$, and applied to each of the atom coordinates in the structure to ensure structural features, such as backbone or sidechain torsion angles, are also corrupted. The model is then tasked with predicting either the per-node noise or the original uncorrupted coordinates. For torsional denoising, the noise is applied to the backbone torsion angles and Cartesian coordinates are recomputed using pNeRF \citep{AlQuraishi2019} and the uncorrupted bond lengths and angles prior to feature computation. Similarly to the coordinate denoising task, the model is then tasked with predicting either the per-residue angular noise or the original dihedral angles.

\textbf{Sequence-Structure Co-Denoising. } This is a multitask formulation of the previously described structure and sequence denoising tasks, with separate output heads for denoising each modality.

\textbf{Masked Attribute Prediction. } 
We use inverse folding (Section \ref{sec:inverse-folding}) as a pretraining task.
The benchmark additionally incorporates the distance, angle and dihedral angle masked-attribute prediction \citep{zhang2023protein} as well as a backbone dihedral angle prediction task.

\textbf{pLDDT Prediction. } Structure prediction models typically provide per-residue pLDDT (predicted Local Distance Difference Test) scores as local confidence measures which have been shown to correlate well with disordered regions \citep{wilson2022alphafold2}. We formulate a node-level regression task on predicted structures, somewhat analogous to structure quality assessment, where the model is tasked with predicting the scaled per-residue pLDDT $y \in [0, 1]$ values.

%%%

\subsection{Downstream Tasks}
We curate several structure-based and sequence-based datasets from the literature and existing benchmarks\footnote{To retain focus on \emph{protein} representation learning, we deliberately exclude commonly-used tasks based on protein-small molecule interactions as it is hard to disentangle the effect of the small molecule representation and the potential for bias \citep{Boyles2019}}, summarised in Table \ref{tab:datasets}. The tasks are selected to evaluate not only the \emph{global} structure representational power of each method, but also to evaluate the ability of each method to learn informative \emph{local} representations for residue-level prediction and annotation tasks.

The raw structures are, where possible and accounting for obsolescence, retrieved directly from the PDB (or another structural source) as several processed datasets used by the community discard full atomic coordinates in favour of retaining only $C_\alpha$ positions, making them unsuitable for in-depth experimentation. 
This provides an entry point for users to apply a custom sequence of pre-processing steps, such as deprotonation or fixing missing regions which are common in experimental data.

\begin{table*}[!t]
    \centering
    \caption{\textbf{Overview of supervised tasks and datasets.}}
    \begin{adjustbox}{max width=\linewidth}
        \begin{tabular}{llcccccc}
        \toprule
        & \textbf{Task} & \textbf{Dataset Origin} & \textbf{Structures} &  \textbf{\# Train} & \textbf{\# Validation} & \textbf{\# Test} & \textbf{Metric} \\
        \midrule
        \multirow{4}{*}{\rotatebox[origin=c]{90}{Node-level}} &
        Inverse Folding & \citet{NEURIPS2019_f3a4ff48} & Experimental
        &
        3.9 M
        &
        105 K
        & 
        180 K & Perplexity
        \\
        & PPI Site Prediction & \citet{gainza2020deciphering} & Experimental 
        &
        478 K
        & 
        53 K
        &
        117 K & AUPRC
        \\
        & Metal Binding Site Prediction & & Experimental
        &
        1.1 M
        &
        13.7 K
        &
        29.8 K & Accuracy
        \\
        & Post-Trans. Mod. Site Prediction &
        \citet{Yan2023} & Predicted 
        
        &
        44 K
        &
        2.4 K
        &
        2.5 K & ROC-AUC \\
        \midrule
        \multirow{4}{*}{\rotatebox[origin=c]{90}{Graph-level}}
        & Fold Prediction & \citet{hou2017} & Experimental 
        &
        12.3 K
        &
        0.7 K
        &
        1.3/0.7/1.3 K & Accuracy
        \\
        & Gene Ontology Prediction & \citet{Gligorijevi2021} & Experimental 
        &
        27.5 K
        &
        3.1 K
        &
        3.0 K & F$_{\text{max}}$
        \\
        & Reaction Class Prediction & \citet{hermosilla2020intrinsic} &  Experimental 
        &
        29.2 K
        &
        2.6 K
        &
        5.6 K & Accuracy
        \\
        & Antibody Dev. Prediction & \citet{NEURIPSDATASETSANDBENCHMARKS2021_4c56ff4c} & Experimental 
        &
        1.7 K
        &
        0.24 K
        &
        0.48 K & AUPRC
        \\
        \bottomrule
        \end{tabular}
    \end{adjustbox}
    \label{tab:datasets}
\end{table*}

%%%

\subsubsection{Node-level Tasks}

\textbf{Inverse Folding. }
\label{sec:inverse-folding} 
Many generative methods for protein design produce backbone structures that require the design of an associated sequence. As a result, inverse folding is an important part of \emph{de novo} design pipelines for proteins \citep{Dauparas2022}.
% This task is a common protein engineering task where the goal is to recover an amino acid sequence given a structure up to backbone completeness. 
Formally, this is a node-level classification task where the model learns a mapping for each residue $r_i$ to an amino acid type $y \in \{1, \dots, n \}$, where $n$ is the vocabulary size ($n=20$ for the canonical set of amino acids).
Inverse folding is a generic task that can be applied to any dataset in the benchmark. In the literature, it is commonly evaluated on the CATH dataset (Section \ref{sec:pre-train-data}) compiled by \citet{NEURIPS2019_f3a4ff48}.

\textbf{PPI Site Prediction. } 
Identifying protein-protein interaction sites has important applications in developing refined protein-protein interaction networks and docking tools, providing biological context to guide protein engineering and target identification in drug discovery campaigns \citep{Jamasb2021}.
This task is a node-level binary classification task where the goal is to predict whether or not a residue is involved in a protein-protein interaction interface.
We use the dataset of experimental structures curated from the PDB by \citet{gainza2020deciphering} and retain the original splits, though we modify the labelling scheme to be based on inter-atomic proximity (3.5 \AA), which can be user-defined, rather than solvent exclusion. The dataset is curated from the PDB by preprocessing such as the presence of one of the seven specified ligands (e.g., ADP or FAD), clustering based on 30\% sequence identity and random subsampling. It contains 1,459 structures, which are randomly assigned to training (72\%), validation (8\%) and test set (20\%). 12 (\AA) radius patches were extracted from the generated structures and a patch labelled as part of a binding pocket if its centre point was < 3 (\AA) away from an atom of the corresponding ligand.

\textbf{Metal Binding Site Prediction. } 
Many proteins coordinate transition metal ions to carry out their functions. Predicting the binding sites of metal ions can elucidate the role of metal binding on protein function.
This is a binary node classification task where each residue is mapped to a label $y \in \{0, 1\}$ indicating whether the residue (or its constituent atoms) is within 3.5 (\AA) of a user-defined metal ion or ligand heteroatom, respectively.
We provide tooling to curate a dataset of experimental structures from the PDB for this task, where binding site assignments for each residue are computed on-the-fly. While the benchmark supports this task on arbitrary subsets of the PDB and ligands, we provide the Zinc-binding dataset from \citet{Drr2023} specifically for this task. The dataset is constructed by sequence-based clustering of the PDB at 30\% sequence identity to remove sequence and structural redundancy. Clusters with a member shorter than 3000 residues, containing at least one zinc atom with resolution better than 2.5 (\AA) determined by x-ray crystallography and not containing nucleic acids are used to compose the dataset. If multiple structures fulfil these criteria, the highest resolution structure is used. The train (2,085) / validation (26) / test (59) splits are constructed such that proteins in the validation and test sets have no partial overlap with any protein in the training data.


\textbf{Post-Translational Modification Site Prediction. } 
Identifying the precise sites where post-translational modifications (PTMs) occur is essential for understanding protein behaviour and designing targeted therapeutic interventions.
We frame prediction of PTM sites as a multilabel classification task where each residue is mapped to a label $y \in \{1, \dots, 13\}$ distinguishing between modifications on different amino acids (e.g. phosphorylation on S/T/Y and N-linked glycosylation on N).
We use a dataset of 48,811 AlphaFold2-predicted structures curated by \citet{Yan2023}, where each structure contains the PTM metadata necessary to construct residue-wise site prediction labels. The dataset is split into training (43,907, validation (2,393) and test (2,511) sets based on 50\% sequence identity and 80\% coverage.

\subsubsection{Graph-level Tasks}

\textbf{Fold Prediction. }
The utility of this task is that it serves as a litmus test for the ability of a model to distinguish different structural folds. It stands to reason that models that perform poorly on distinguishing fold classes likely learn limited or low-quality structural representations.
This is a multiclass graph classification task where each protein, $\mathcal{G}$, is mapped to a label $y \in \{1, \dots, 1195\}$ denoting the fold class.
We adopt the fold classification dataset originally curated from SCOP 1.75 by \citep{hou2017}. This provides three different test sets stratified based on topological similarity: Fold, in which proteins originating from the same superfamily are absent during training; Superfamily, in which proteins originating from the same family are absent during training; and Family, in which proteins from the same family are present during training.

\textbf{Gene Ontology Prediction. }
Predicting protein function in the form of functional annotations such as GO terms has important applications in protein analysis and engineering, providing researchers with the ability to cluster functionally-related structures or to guide protein generation methods to design new proteins with desired functional properties.
This is a multilabel classification task, assigning functional Gene Ontology (GO) annotation to structures. GO annotations are assigned within three ontologies: biological process (BP), cellular component (CC) and molecular function (MF). We use the dataset of experimental structures curated from the PDB by \citet{Gligorijevi2021} and retain the original multi-cutoff based splits, with cutoff at 95\% sequence similarity. 

\textbf{Reaction Class Prediction. } 
As proteins' reaction classifications are based on their enzyme-catalyzed reaction according to all four levels of the standard Enzyme Commission (EC) number, methods that predict such classifications may help elucidate the function of newly-designed proteins as they are developed.
This is a multiclass graph classification task where each protein, $\mathcal{G}$, is mapped to a label $y \in {\{1, ..., 384\}}$ denoting which class of reactions a given protein catalyzes; all four levels of the EC assignment are employed to define the reaction class label.
We adopt the reaction class prediction dataset originally curated from the PDB by \citet{hermosilla2020intrinsic}, split on the basis of sequence similarity using a 50\% threshold.

\textbf{Antibody Developability Prediction. }
Therapeutic antibodies must be optimised for favourable physicochemical properties in addition to target binding affinity and specificity to be viable development candidates. Consequently, we frame prediction of antibody developability as a binary graph classification task indicating whether a given antibody is developable.We adopt the antibody developability dataset originally curated from SabDab \citep{dunbar2014sabdab} by \citet{Chen2020}.
This dataset contains 2,426 antibodies that have both sequences and PDB structures available, where each example contains both a heavy chain and a light chain with resolution < 3 (\AA). 
The label is based on thresholding the developability index (DI) \citep{Lauer2012} 
as computed by BIOVIA's platform \citep{Accelrys2018BioviaDiscoveryStudio}, which relies on an antibody's hydrophobic and electrostatic interactions.
This task is interesting from a benchmarking perspective as it enables targeted performance assessment of models on a specific (immunoglobulin) fold, providing insight into whether general-purpose structure-based encoders can be applicable to fold-specific tasks.

\subsection{Pre-training Datasets}\label{sec:pre-train-data}

The benchmark contains several large corpora of both experimental and predicted structures that can be used for pretraining or inference. We provide utilities for configuring supervised tasks and splits directly from the PDB.
Additionally, we build storage-efficient dataloaders for large pretraining corpora of predicted structures (AlphaFoldDB, ESM Atlas).
We believe our codebase will considerably reduce the barrier to entry for working with large structure-based datasets. 
% Additionally, we provide ready-to-go dataloaders for several large-scale collections of predicted structures derived from both AlphaFold2 \citep{jumper2021highly} and ESMFold \citep{lin2022language}. 
% This is facilitated by FoldComp \citep{Kim2023}, a (minimally) lossy compression scheme for predicted protein structures. FoldComp stores protein structures as a collection of discretised dihedral and bond angles which can be used to reconstruct the whole structure using fixed bond lengths and canonical amino acid geometry. FoldComp achieves a disk-space reduction of almost an order of magnitude, describing a residue with only 13 bytes -- down from 97 bytes per-residue in a traditional uncompressed format. Whilst lossy, this procedure results in 0.08 \AA\ and 0.14 \AA\ RMSD for backbone and all-atom reconstruction, making it highly suitable for pretraining tasks which use input representations complete up to the backbone. Furthermore, this lightweight format enables the dataloaders in the benchmark to read structures \emph{directly from disk} with no pre-processing or caching required.

\subsubsection{Experimental Structures}

\textbf{PDB. } We provide utilities for curating datasets directly from the Protein Data Bank \citep{Berman2000}. In addition to using the collection in its entirety, users can define filters to subset and split the data using a combination of structural similarity, sequence similarity or temporal strategies. Structures can be filtered by length, number of chains, resolution, deposition date, presence/absence of particular ligands and structure determination method. 
% The benchmark supports working with PDB structures in both \texttt{.pdb} and \texttt{.mmtf} format \citep{Bradley2017}, which significantly reduces the requirements for data storage.

\textbf{CATH. } We provide the dataset derived from CATH 4.2 40\% \citep{Knudsen2010} non-redundant chains developed by \citet{NEURIPS2019_f3a4ff48} as an additional, smaller, pretraining dataset. 
% These data are split based on random assignment of the CATH topology classifications based on an 80/10/10 split.

\textbf{ASTRAL. } ASTRAL \citep{Brenner2000} provides protein \emph{domain} structures which are regions of proteins that can maintain their structure and function independently of the rest of the protein. Domains typically exhibit highly-specific functions and can be considered structural building blocks.

\subsubsection{Predicted Structures}

\textbf{AlphaFoldDB Representative Structures.} This dataset contains 2.27 million representative structures, identified through large-scale structural-similarity-based clustering of the 214 million structures contained in the AlphaFold Database \citep{Varadi2021} using FoldSeek \citep{vanKempen2023}. We additionally provide a subset of this collection --- the so-called dark proteome --- corresponding to the 31\% of the representative structures that lack annotations.

\textbf{ESM Atlas, ESM High Quality.} These datasets are compressed collections of predicted structures produced by ESMFold. ESM Atlas is the full collection of all 772m predicted structures for the MGnify 2023 release \citep{Richardson2022}. ESM High Quality is a curated subset of high confidence (mean pLDDT) structures from the collection.

\section{Methods and Experimental Setup}
\begin{itemize}
    \item Paragraph justifying our choices of a subset of tasks and models
\end{itemize}


\paragraph{Architectures}
We evaluate X geometric GNN architectures, spanning the range of message passing body order and tensor order. 


\paragraph{Pre-training Dataset} For all pre-training tasks we use the \texttt{afdb\_rep\_v4} dataset compiled by \citet{BarrioHernandez2023}. This dataset contains 2.27 million representative structures, identified through large-scale structural-similarity based clustering of the 214 million structures contained in the AlphaFold Database \cite{Varadi2021} using FoldSeek \cite{vanKempen2023}. This dataset therefore provides a rich diversity of protein structures and is substantially larger than any other previously used structure-based pre-training corpus that we are aware of, whilst remaining of a size that is amenable to experimentation.

\paragraph{Featurisation Schemes}
The benchmark includes comprehensive featurisation schemes for both scalar and vector-valued feature computation. 


\paragraph{Noising Schemes} For structure-based denoising we draw noise samples from a guassian distribution and scale by 0.1. For structure-based denoising, we use the mutation strategy and corrupt 25\% of the residues in each protein. When denoising is used as an auxillary task, we weight the loss with a coefficient $\lambda = 0.1$, similar to NoisyNodes \cite{godwin2021simple}.

\paragraph{Training}
We use a ReduceLRonPlateau learning rate scheduler for the downstream tasks with a patience of 5 epochs and a reduction factor of 0.6. For the pre-training tasks, we use a linear warmup with cosine decay schedule.



\subsection{Baselines}
\section{Results}
%\subsection{Baseline Results without Pre-training}
\subsection{Auxiliary Tasks Consistently Improve Performance Over Baselines}

We first set out to determine (1) whether invariant or equivariant models perform better on our set of tasks, (2) which input representation is the best for each respective task and (3) whether auxiliary denoising tasks improve model performance. Table \ref{tab:baseline_graph_classification_results} shows that (1) models that perform message passing on line graphs of protein structures (e.g., GearNet-Edge) produce the best representations for fold classification, whereas equivariant models produce the best representations for protein-protein interaction site prediction. (2) Notably, featurising models with C$\alpha$ atoms and virtual angles provide the best performance overall. The same set of results also suggests that (3) both structure denoising and particularly sequence denoising are useful auxiliary tasks for training protein structure encoders, with the exception of inverse folding.

\begin{table}[!ht]

\caption{Baseline tasks without pre-training. Results are given as: \colorbox{orange!20}{no auxiliary task} / \colorbox{blue!20}{+sequence denoising} / \colorbox{green!20}{+structure denoising}. Coloured boxes mark the best auxiliary tasks per method and featurisation, underline the best featurisation per method and bold the best method, all on a per-task basis. Greyed cells denote invalid task-setup combinations (e.g. inverse folding and sequence denoising as auxiliary task). Denoising auxiliary tasks consistently improve performance across architectures. Lines denote configurations that failed to converge after 6 hours.} 
\label{tab:baseline_graph_classification_results}

\begin{adjustbox}{max width=\linewidth}
\begin{tabular}{cllcccccc|cccllll}
\toprule

\multirow{2}{*}{\textbf{Method}} & \multicolumn{1}{c}{\multirow{2}{*}{\textbf{Features}}} & \multicolumn{1}{c}{} & %\multicolumn{1}{c}{\multirow{2}{*}{\textbf{EC} ($\uparrow$)}}
& & \multicolumn{3}{c}{\textbf{Fold} ($\uparrow$)} & &
%\multirow{2}{*}{\textbf{PTM} ($\uparrow$)} 
& \multirow{2}{*}{\textbf{PPI Site} ($\uparrow$)} & \multirow{2}{*}{\textbf{Inverse Folding} ($\downarrow$)} \\
\cmidrule{6-8}
 & \multicolumn{1}{c}{} &  & \multicolumn{1}{c}{} & \multicolumn{1}{c}{} & \textbf{Fold} & \textbf{Family} & \textbf{Superfamily} &  &  &  &  &  &  &  \\
\midrule 
\multirow{5}{*}{SchNet} & \caa &  & 
%49.70 / \colorbox{blue!20}{57.45} / 44.50 
&  & 12.23 / \colorbox{blue!20}{14.89} / 12.98 & \colorbox{orange!20}{68.75} / 65.50 / 65.15 & 16.57 / \colorbox{blue!20}{17.28} / 15.80  &  &  & \colorbox{orange!20}{95.52} /  94.82 & \cellcolor{gray!20} &  &  &  \\

 & \caa + Seq. &  & 
 %46.44 / \colorbox{blue!20}{53.83} / 45.25 
 &  & \colorbox{orange!20}{16.01} / 14.41 / 14.50 & \colorbox{orange!20}{62.45} / 60.58 / 61.95 &  18.11 / \colorbox{blue!20}{18.51} / 14.64 &  &  & 95.51 / \colorbox{green!20}{\underline{95.57}} & \colorbox{orange!20}{12.07} / 12.32 &  &  &  \\
 
 & \caa + \virt &  & 
 %\colorbox{orange!20}{55.56} / -------- / 50.90 
 &  & 14.90 / \colorbox{blue!20}{16.98} / 15.98 & 71.58 / \colorbox{blue!20}{74.17} / 71.46 & 21.56 / \colorbox{blue!20}{23.39} / 20.44 &  &  &  95.43 / \colorbox{green!20}{95.59} & \colorbox{orange!20}{\underline{11.15}} / 11.46 &  &  &  \\
 
 & \caa + \virt + \bb &  &
 %54.81 / \colorbox{blue!20}{\underline{61.23}} / 54.81
 &  & 15.99 / \colorbox{blue!20}{\underline{19.50}} / 16.86 & 72.75 / \colorbox{blue!20}{\underline{75.17}} / 72.55 & \colorbox{orange!20}{\underline{24.44}} / 22.89 / 23.50 &  &  & 95.42 / \colorbox{green!20}{95.56} & \colorbox{orange!20}{11.20} / 11.41 &  &  &  \\
 
 & \caa + \virt + \bb + \schi &  & 
 %55.21 / \colorbox{blue!20}{57.89} / -------- 
 &  & 13.01 / \colorbox{blue!20}{15.43} / 14.36 & 71.54 / 71.39 / \colorbox{green!20}{71.55} & 22.31 / 21.92 / \colorbox{green!20}{23.18} &  &  & \colorbox{orange!20}{95.47} / 95.46 & \cellcolor{gray!20} &  &  &  \\
 
\midrule
\multicolumn{1}{l}{\multirow{5}{*}{DimeNet}} & \caa &  &
%-------- / -------- / 27.99 
& & 16.34 / \colorbox{blue!20}{18.37} / 15.06 & 71.87 / \colorbox{blue!20}{74.08} / 55.37 & 21.92 / \colorbox{blue!20}{23.36} /15.47 &  &  & 95.54 / \colorbox{green!20}{\underline{95.66}} &  \cellcolor{gray!20}&  &  &  \\

\multicolumn{1}{l}{} & \caa + Seq &  & 
%-------- / -------- / 17.72 
&  & 16.30 / \colorbox{blue!20}{20.31} / 15.18 & 66.64 / \colorbox{blue!20}{72.99} / 48.42 & 20.36 / \colorbox{blue!20}{25.32} / 13.54 &  &  & 95.49 / \colorbox{green!20}{95.61}  & 10.64 / -------- &  &  &  \\

\multicolumn{1}{l}{} & \caa + \virt &  & 
%-------- / -------- / 30.99
&  & \colorbox{orange!20}{18.14} / 16.36 / 16.14 & \colorbox{orange!20}{70.45} / 62.38 / 62.06 & \colorbox{orange!20}{21.04} / 19.51 / 17.79 & &  & \colorbox{orange!20}{95.53} / 95.45 & \colorbox{orange!20}{10.18} /  11.23 &  &  &  \\

\multicolumn{1}{l}{} & \caa + \virt + \bb &  & 
%-------- / -------- / 31.00
&  & 
18.39 / \colorbox{blue!20}{\underline{21.65}} / -------- & 72.94 / \colorbox{blue!20}{\underline{77.14}} / -------- & 23.23 / \colorbox{blue!20}{\underline{25.36}} / -------- &  &  &  95.52 / \colorbox{green!20}{95.60}  & \underline{\textbf{9.91}} / -------- &  &  \\

\multicolumn{1}{l}{} & \caa + \virt + \bb + \schi &  & 
%-------- / -------- / -------- 
&  &  16.83 / \colorbox{blue!20}{19.73} / -------- & 69.67 / \colorbox{blue!20}{75.59} / -------- & 22.03 / \colorbox{blue!20}{23.88} / -------- &  &  & 95.42 / \colorbox{green!20}{95.52} &  \cellcolor{gray!20} &  &  &  \\
\midrule

\multicolumn{1}{l}{\multirow{5}{*}{GearNet-Edge}} & \caa &  &  &  & 32.45 / 33.01 / \colorbox{green!20}{33.59} & 95.03 / \colorbox{blue!20}{95.45} / 94.74 & 45.59 /\colorbox{blue!20}{48.44} / 48.28  & & & 94.62 / \colorbox{green!20}{95.27}  &  \cellcolor{gray!20} &  & \\

\multicolumn{1}{l}{} & \caa + Seq &  &  &  & 30.02 / 30.85 / \colorbox{green!20}{32.14} & \colorbox{orange!20}{94.84} / 93.88 / 94.74  & 45.65 / \colorbox{blue!20}{46.86} / 46.69  &  &  &  94.71 /  \colorbox{green!20}{95.47} & -------- / --------   &  &  &  \\

\multicolumn{1}{l}{} & \caa + \virt &  &  &  &27.86 / \colorbox{blue!20}{\textbf{\underline{34.86}}} / 33.48 & 91.16 / \colorbox{blue!20}{\textbf{\underline{96.21}}} / 95.51  & 43.10 / \colorbox{blue!20}{\textbf{\underline{49.81}}} / 48.51  &  &  & 94.70 / \colorbox{green!20}{95.59}  & 12.23 / --------  \\

\multicolumn{1}{l}{} & \caa + \virt + \bb &  &  &  &  29.68 / 29.98 / \colorbox{green!20}{31.90} & 93.56 / 92.79 / \colorbox{green!20}{95.46} & 44.78 / 45.71 / \colorbox{green!20}{47.27}  & & & 94.94 / \colorbox{green!20}{95.59} & 12.28 / -------- &  &  \\

\multicolumn{1}{l}{} & \caa + \virt + \bb + \schi &  &  &  & -------- / -------- / -------- & -------- / -------- / -------- & -------- / -------- / -------- &  &  &  -------- / --------  & \cellcolor{gray!20}  &  &  &   \\
\midrule



\multicolumn{1}{l}{\multirow{5}{*}{EGNN}} & \caa &  &  &  & 13.55 / \colorbox{blue!20}{14.08} / -------- & \colorbox{orange!20}{71.80} / 65.82 / --------& \colorbox{orange!20}{18.68} / 14.08 / --------&  &  & 95.89 /\colorbox{green!20}{95.98} & \cellcolor{gray!20} &  &  &  \\

\multicolumn{1}{l}{} & \caa + Seq &  &  &  & 12.64 / \colorbox{blue!20}{12.77} / -------- & \colorbox{orange!20}{68.70} / 59.31 / -------- & \colorbox{orange!20}{15.86} / 11.66  / --------&  &  & 96.27 / \colorbox{green!20}{\underline{96.30}}  & \colorbox{orange!20}{11.93} / 12.05 &  &  &  \\

\multicolumn{1}{l}{} & \caa + \virt &  &  &  & 23.07 / \colorbox{blue!20}{24.73} / -------- & \colorbox{orange!20}{\underline{90.06}} / 89.35 / -------- & 33.19 / \colorbox{blue!20}{35.39} / -------- &  &  & \colorbox{orange!20}{96.13} /  95.94 &  \colorbox{orange!20}{\underline{11.44}} / 11.46 &  &  &  \\

\multicolumn{1}{l}{} & \caa + \virt + \bb &  &  &  & 22.84 / \colorbox{blue!20}{\underline{25.63}} / -------- & 88.11 / \colorbox{blue!20}{89.62} / -------- & 33.83 / \colorbox{blue!20}{\underline{36.95}} / -------- &  &  & 95.89 / \colorbox{green!20}{96.25} & \colorbox{orange!20}{11.52} / 11.75 &  &  &  \\

\multicolumn{1}{l}{} & \caa + \virt + \bb + \schi &  &  &  & \colorbox{orange!20}{26.51} / 23.00 / --------& \colorbox{orange!20}{89.60} / 84.47 / -------- & 31.92 / \colorbox{blue!20}{35.27} / -------- &  &  &  95.79 / \colorbox{green!20}{96.16} & \cellcolor{gray!20} &  &  &   \\
\midrule


\multicolumn{1}{l}{\multirow{5}{*}{GCPNet}} & \caa &  & 
%/ -------- / 49.72 
&  & 19.65 / \colorbox{blue!20}{28.79} / -------- & 76.73 / \colorbox{blue!20}{88.47} / -------- & 21.93 / \colorbox{blue!20}{33.31} / -------- &  & & 96.39 /\colorbox{green!20}{96.49} & \cellcolor{gray!20} &  &  &  &  \\

\multicolumn{1}{l}{} & \caa + Seq &  & 
%/ -------- / 51.77
&  & 23.54 / \colorbox{blue!20}{\underline{29.84}} / -------- & 81.23 / \colorbox{blue!20}{89.84} / -------- & 25.43 / \colorbox{blue!20}{36.82} / -------- &  &  & \colorbox{orange!20}{96.55} / 96.53 &  -------- / --------   &  &  &  \\

\multicolumn{1}{l}{} & \caa + \virt &  &  &  & 24.96 / \colorbox{blue!20}{29.64} / --------  & 87.67 / \colorbox{blue!20}{91.63} / -------- & 33.71 / \colorbox{blue!20}{\underline{40.62}} / -------- &  &  & 96.46 / 96.46 &  -------- /  --------   &  &  &  \\

\multicolumn{1}{l}{} & \caa + \virt + \bb &  &  &  & 27.62 / \colorbox{blue!20}{29.19} / -------- & 88.13 / \colorbox{blue!20}{\underline{91.72}} / -------- & 35.68 / \colorbox{blue!20}{39.18} / -------- &  &  &  96.45 / \colorbox{green!20}{\underline{\textbf{96.61}}} &  -------- / --------   &  &  &  \\

\multicolumn{1}{l}{} & \caa + \virt + \bb + \schi &  &  &  & -------- & -------- & -------- &  &  & -------- / -------- & \cellcolor{gray!20} & &  &    \\

\midrule
\multicolumn{1}{l}{\multirow{5}{*}{TFN}} & \caa &  &  &  &  &  &  &  &  &  &  &  &  &  \\
\multicolumn{1}{l}{} & \caa + Seq &  &  &  &  &  &  &  &  &  &  &  &  &  \\
\multicolumn{1}{l}{} & \caa + Virt. Angles &  &  &  &  &  &  &  &  &  &  &  &  &  \\
\multicolumn{1}{l}{} & Backbone &  &  &  &  &  &  &  &  &  &  &  &  &  \\
\multicolumn{1}{l}{} & All atom &  &  &  &  &  &  &  &  &  &  &  &  &   \\
\midrule
\multicolumn{1}{l}{\multirow{5}{*}{MACE}} & \caa &  &  &  &  &  &  &  &  &  &  &  &  &  \\
\multicolumn{1}{l}{} & \caa + Seq &  &  &  &  &  &  &  &  &  &  &  &  &  \\
\multicolumn{1}{l}{} & \caa + Virt. Angles &  &  &  &  &  &  &  &  &  &  &  &  &  \\
\multicolumn{1}{l}{} & Backbone &  &  &  &  &  &  &  &  &  &  &  &  &  \\
\multicolumn{1}{l}{} & All atom &  &  &  &  &  &  &  &  &  &  &  &  &   \\

\bottomrule
\end{tabular}


\end{adjustbox}
\end{table}



%\subsection{Graph Classification tasks with denoising auxiliary tasks}

%\subsection{Pre-training via denoising}
\subsection{Incorporating More Structural Detail Improves Pre-Training Performance}

\begin{table}[!ht]
    \centering
    \caption{Validation performance for pre-training tasks on \texttt{afdb\_rep\_v4}. Incorporating backbone geometry consistently improves pre-training performance. \textbf{Inverse Folding}: perplexity, \textbf{pLDDT, Structure Denoising, Torsional Denoising}: RMSE, \textbf{Seq. Denoising}: Accuracy.}. 
    \label{tab:pre-training}
    \begin{adjustbox}{max width=\linewidth}
    \begin{tabular}{ccccccccccccccccccc}
         \toprule
         & \multirow{2}{*}{\bf{Method}} & &
            \multicolumn{5}{c}{\bf{Task}}& &
            \\
            \cmidrule{4-8}
             & & & \bf{Inverse Folding} ($\downarrow$) & \bf{pLDDT Pred.} ($\downarrow$) & \bf{Structure Denoising} ($\downarrow$)& \bf{Seq. Denoising} ($\uparrow$) & \bf{Torsional Denoising} ($\downarrow$) \\
         \midrule
         \multirow{5}{*}{\rotatebox{90}{\small {$C_\alpha$ + \virt }}} 
         & SchNet & & 7.791 & 0.2397 & 0.0704 & 36.81 & \underline{0.0586}\\
         & DimeNet & & \textbf{6.016} &  \textbf{0.2100} & \textbf{0.0655} & \textbf{47.07} & -------- \\
         & GearNet-Edge & & 6.596 & 0.2326 & \underline{0.0672} & 43.76 & 0.0615 \\
         & EGNN & & \textbf{6.016} & 0.2406 & 0.0700 & 40.51 & \underline{0.0586}\\
         & GCPNet & & 6.243 & 0.2395 & 0.0679 & \underline{44.81} & \textbf{0.0562}\\
         \midrule
         \multirow{5}{*}{\rotatebox{90}{\small {$C_\alpha + \phi, \psi, \omega$ }}} 
         & SchNet & & 5.562 & \underline{0.2388} & 0.0603 & 45.61 & 0.0489\\
         & DimeNet & & 5.962 & \textbf{0.2094} & \textbf{0.0543} & 46.70 & -------- \\
         & GearNet-Edge & & \underline{5.324} & 0.2402 & 0.0562 & 50.15 & 0.0538 \\
         & EGNN & & 5.962 & 0.2403  & 0.0593 & \underline{53.80} & \underline{0.0487}\\
         & GCPNet & & \textbf{3.839} & 0.2399 & \underline{0.0561} & \textbf{59.54} & \textbf{0.0443} \\
        \bottomrule
         
    \end{tabular}
    \end{adjustbox}   
\end{table}

Since auxiliary tasks improved performance, we then investigated protein structure pre-training to find out (1) which input representation is best for pre-training and (2) which models benefit from which pre-training task. Table \ref{tab:pre-training} shows that (1) incorporating dihedral angles consistently improves validation metrics on pre-training tasks, more so than architecture. These results also suggest that (2) inverse folding, sequence denoising, and torsional denoising benefit equivariant models the most in the context of pre-training, whereas pLDDT prediction and structure denoising benefit invariant models the most, suggesting that certain pre-training tasks benefit certain classes of models more than other tasks.

%\subsection{Results with Pre-trained Models}
\subsection{Pre-training Helps but Incorporating More Structural Detail Negatively Affects Downstream Performance}

\begin{table}[!ht]
\caption{Pre-trained model results on graph classification (left) and node classification (right) tasks. Results are given as: \colorbox{blue!20}{sequence denoising} / \colorbox{green!20}{structure denoising} / \colorbox{purple!20}{torsion denoising} (except for inverse folding, which is pre-trained with an inverse folding task). Coloured boxes mark the best
pre-training tasks per method and featurisation, underline the best featurisation per method and bold
the best method, all on a per-task basis. Equivariant models benefit the most from pre-training and interestingly perform best using virtual torsion and bond angles as input features. Lines denote configurations that failed to converge after 6 hours.}
\label{tab:pre_trained_graph_classification_results}

\begin{adjustbox}{max width=\linewidth}
\begin{tabular}{cllccccc|ccclllll}

\toprule

\multirow{2}{*}{\textbf{Method}} & \multicolumn{1}{c}{\multirow{2}{*}{\textbf{Features}}} & \multicolumn{1}{c}{} & 
%\multicolumn{1}{c}{\multirow{2}{*}{\textbf{EC} ($\uparrow$)}} 
& & \multicolumn{3}{c}{\textbf{Fold} ($\uparrow$)} & 
%\multirow{2}{*}{\textbf{PTM} ($\uparrow$)} 
& \multirow{2}{*}{\textbf{PPI Site} ($\uparrow$)} & \multirow{2}{*}{\textbf{Inverse Folding} ($\downarrow$)}  \\
\cmidrule{6-8}
 & \multicolumn{1}{c}{} &  & \multicolumn{1}{c}{} & \multicolumn{1}{c}{} & Fold & Family & Superf. &  &  &  &  &  &  &  \\
\midrule 
\multirow{2}{*}{SchNet} & \caa + \virt &  &
%58.68 / -------- / 57.83  
&  & 18.01 / 18.16 / \colorbox{purple!20}{\underline{25.94}}  & 73.65 / 74.43 / \colorbox{purple!20}{\underline{88.76}} & 23.17 / 21.09 / \colorbox{purple!20}{\underline{38.41}} &  & -------- / \colorbox{green!20}{95.34} / 95.24 & 9.67  \\

 & \caa + \virt + \bb &  & 
 %56.54 / -------- / --------  
 &  &  16.63 / \colorbox{green!20}{18.25} / 17.75  & 69.5 / 70.40 /  \colorbox{purple!20}{73.22}  & 21.17 / \colorbox{green!20}{23.50} / 23.39 & & -------- / -------- / -------- & 10.72  \\
 \midrule
%\multicolumn{1}{l}{\multirow{2}{*}{DimeNet}} & \caa + \virt &  &  &  & -------- / -------- / -------- & -------- / -------- / --------& -------- / -------- /--------  & & -------- / -------- / --------&  --------  &  &  &  \\

%\multicolumn{1}{l}{} & \caa + \virt + \bb &  &  & & -------- / -------- / -------- & -------- / -------- / --------& -------- / -------- /-------- &  & -------- / -------- / -------- & -------- &  &  &  &  \\

\midrule
\multicolumn{1}{l}{\multirow{2}{*}{GearNet-Edge}} & \caa + \virt &  &  &  & \colorbox{blue!20}{\underline{32.25}} / 31.06 / 31.06 & \colorbox{blue!20}{\underline{94.10}} / 92.00 / 90.80 &  \colorbox{blue!20}{\underline{44.95}} / 44.93 / 42.55 &  & \colorbox{blue!20}{\underline{95.49}} / 94.47 / 94.99 & 7.73 \\

\multicolumn{1}{l}{} & \caa + \virt + \bb &  &  &  & 28.65 / \colorbox{green!20}{29.79} / 28.77  & 81.82 / \colorbox{green!20}{91.39} / 89.26 & 36.79 / \colorbox{green!20}{41.62} / 39.26 & & \colorbox{blue!20}{95.10} / 94.58 / 94.54  &  8.56 \\

\midrule
\multicolumn{1}{l}{\multirow{2}{*}{EGNN}} & \caa + \virt &  &  &  &  \underline{26.53} / -------- / -------- & \underline{93.19} / -------- / --------  & \underline{33.76} / -------- / --------&  & \colorbox{blue!20}{\underline{96.65}} / 96.29 / 96.29 & 8.71  \\

\multicolumn{1}{l}{} & \caa + \virt + \bb &  &  & & 21.75 / -------- / -------- & 86.76 / -------- / --------& 30.04 / -------- / -------- &  & 96.02 / \colorbox{green!20}{96.08} / 95.71 & 9.59 & &  &  &  \\

\midrule
\multicolumn{1}{l}{\multirow{2}{*}{GCPNet}} & \caa + \virt &  &  &  & 33.81 / \colorbox{green!20}{\textbf{\underline{37.54}}} / -------- & 95.24 / \colorbox{green!20}{\textbf{\underline{96.23}}}  / --------& 46.86 / \colorbox{green!20}{\textbf{\underline{50.14}}} / -------- &  & \colorbox{blue!20}{\textbf{\underline{96.89}}} / 96.33 / -------- & \textbf{7.39} &  &  &  \\

\multicolumn{1}{l}{} & \caa + \virt + \bb  &  &  &  &  30.12 / \colorbox{green!20}{34.81} / -------- & 90.67 / \colorbox{green!20}{93.20} / -------- & 41.61 / \colorbox{green!20}{46.15} / -------- & & \colorbox{blue!20}{96.40} / 96.18 / --------  & --------&  &  \\

\bottomrule
\end{tabular}
\end{adjustbox}
\end{table}

After pre-training and observing that more fine-grained input representations improve pre-training performance, we explore (1) whether these lessons from pre-training translate to downstream tasks and (2) which combination of parameters performs best on downstream tasks. The results in Table \ref{tab:pre_trained_graph_classification_results} suggest that (1) equivariant models benefit the most from pre-training on structure-based tasks. Surprisingly, these results also suggest that virtual torsion and bond angles are the most informative input representation, as including backbone dihedral angles consistently degrades performance despite providing a richer description of structure. This suggests that, compared to backbone dihedral angles, virtual angles may provide useful information that is more easily transferable between AlphaFold-predicted structures and experimental protein structures in the context of pre-training. However, we note that pre-trained results with backbone dihedral angles still yield improved results over non-pretrained baselines. Also interestingly, from the results in Table \ref{tab:pre_trained_graph_classification_results} it appears that (2) structure denoising serves equivariant models such as GCPNet best for performance in downstream tasks.




\section{Future Work}
%Protein representation learning is an exciting field with incredible room for expansion, innovation, and impact. The exponentially growing gap between labeled and unlabeled protein data means that self-supervised learning will continue to play a large role in the future of computational protein modeling. Our results show that no single self-supervised model performs best across all protein tasks. We believe this is a clear challenge for further research in self-supervised learning, as there is a huge space of model architecture, training procedures, and unsupervised task choices left to explore. It may be that language modelling as a task is not enough, and that protein-specific tasks are necessary to push performance past state of the art. Further exploring the relationship between alignment-based and learned representations will be necessary to capitalize on the advantages of each. We hope that the datasets and benchmarks in TAPE will provide a systematic model-evaluation framework that allows more machine learning researchers to contribute to this field.


\section{Societal Impact}
This work focuses on building a comprehensive and multi-task benchmark for protein structure representation learning. In this benchmark, we provide several large pre-training corpuses, featurisation schemes, model implementations and benchmarking tasks to evaluate the effectiveness of protein sequence encoding methods. The variety of tasks can enable us to develop insight into effective pre-training strategies, and whether pre-trained protein structural representations can have material impact in real-world computational biology and drug discovery research activities. It is not lost on us that these models can play a role in developing, for example, harmful chemical matter in the hands of a bad actor. Additionally, the training and pre-training of very large models can contribute to climate change. However, we believe the developing highly effective structural representations will have broad, positive implications across biology and medicine that significantly outweigh the potential for misuse.

%\bibliography
\bibliography{bibliography}



%%%%%%%%%%%%%%%%%%%%%%%%%%%%%%%%%%%%%%%%%%%%%%%%%%%%%%%%%%%%
%\section*{Checklist}

\begin{enumerate}

\item For all authors...
\begin{enumerate}
  \item Do the main claims made in the abstract and introduction accurately reflect the paper's contributions and scope?
    \answerYes{}
  \item Did you describe the limitations of your work?
    \answerYes{}
  \item Did you discuss any potential negative societal impacts of your work?
    \answerYes{}
  \item Have you read the ethics review guidelines and ensured that your paper conforms to them?
    \answerYes{}
\end{enumerate}

\item If you are including theoretical results...
\begin{enumerate}
  \item Did you state the full set of assumptions of all theoretical results?
    \answerNA{}
	\item Did you include complete proofs of all theoretical results?
    \answerNA{}
\end{enumerate}

\item If you ran experiments (e.g. for benchmarks)...
\begin{enumerate}
  \item Did you include the code, data, and instructions needed to reproduce the main experimental results (either in the supplemental material or as a URL)?
    \answerYes{}
  \item Did you specify all the training details (e.g., data splits, hyperparameters, how they were chosen)?
    \answerYes{}
	\item Did you report error bars (e.g., with respect to the random seed after running experiments multiple times)?
    \answerNo{}
	\item Did you include the total amount of compute and the type of resources used (e.g., type of GPUs, internal cluster, or cloud provider)?
    \answerYes{}
\end{enumerate}

\item If you are using existing assets (e.g., code, data, models) or curating/releasing new assets...
\begin{enumerate}
  \item If your work uses existing assets, did you cite the creators?
    \answerYes{}
  \item Did you mention the license of the assets?
    \answerYes{} In the repository
  \item Did you include any new assets either in the supplemental material or as a URL?
    \answerYes{}
  \item Did you discuss whether and how consent was obtained from people whose data you're using/curating?
    \answerNA{}{} Publicly available.
  \item Did you discuss whether the data you are using/curating contains personally identifiable information or offensive content?
    \answerNA{}{}
\end{enumerate}

\item If you used crowdsourcing or conducted research with human subjects...
\begin{enumerate}
  \item Did you include the full text of instructions given to participants and screenshots, if applicable?
    \answerNA{}
  \item Did you describe any potential participant risks, with links to Institutional Review Board (IRB) approvals, if applicable?
    \answerNA{}
  \item Did you include the estimated hourly wage paid to participants and the total amount spent on participant compensation?
    \answerNA{}
\end{enumerate}

\end{enumerate}

%%%%%%%%%%%%%%%%%%%%%%%%%%%%%%%%%%%%%%%%%%%%%%%%%%%%%%%%%%%%
%%%%%%%%%%%%%%%%%%%%%%%%%%%%%%%%%%%%%%%%%%%%%%%%%%%%%%%%%%%%
\end{document}
